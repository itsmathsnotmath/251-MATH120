%%%%%%%%%%%%%%%%%%%%%%%%%%%%%%%%%%%%%%%%%%%%%%%%%%%%%%%%%%%%%%%%
%                                                              %
% This is a LaTeX file.  It is a text file that is compiled    %
% by a program called LaTeX into a pretty PDF file.            %
% If you're viewing this file on Overleaf,                     %
% you'll see that PDF in the window to the right.              %
%                                                              %
% The LaTeX macro language is complicated, so we have inserted %
% lots of documenting comments into the file.  Comments start  %
% with `%' and continue to the end of the line.                %
%                                                              %
% Boxed comments are provided to let you, the student,         %
% understand what that part of the code is doing or to provide %
% you with instructions.                                       % 
%                                                              %
% Skip anything else you don't understand, or ask me.          %
%                                                              %
%%%%%%%%%%%%%%%%%%%%%%%%%%%%%%%%%%%%%%%%%%%%%%%%%%%%%%%%%%%%%%%%
%
\documentclass{article}
\usepackage[left=1in,right=1in,top=1in,bottom=1in]{geometry}
% 
%%%%%%%%%%%%%%%%%%%%%%%%%%%%%%%%%%%%%%%%%%%%%%%%%%%%%%%%%%%%%%%%
%                                                              %
% This is the preamble of the document. This is where we       %
% declare the packages we need for the pdf file to compile     %
% correctly. Packages contain the different commands we        %
% need to use that allow us to format the document nicely.     %
%                                                              %
% You shouldn't need to edit any of these packages.            %
%                                                              %
%%%%%%%%%%%%%%%%%%%%%%%%%%%%%%%%%%%%%%%%%%%%%%%%%%%%%%%%%%%%%%%%
% 
\usepackage{amsmath, amsthm, amsfonts} % these packages contain most of the commands needed to format the maths symbols
\usepackage{enumerate} % package that contains options for the \begin{enumerate} environment
\usepackage{hyperref} % package that allows hyperlinks
\newtheorem*{theorem}{Theorem} % defines an environment with the heading "Theorem". The * supresses the numbering
\theoremstyle{definition}
\newtheorem*{definition}{Definition} % defines an environment with the heading "Definition". The * supresses the numbering
\newtheorem{solution}{Solution} % defines an environment with the heading "Solution".
% 
%%%%%%%%%%%%%%%%%%%%%%%%%%%%%%%%%%%%%%%%%%%%%%%%%%%%%%%%%%%%%%%%
%                                                              %
% In the author command below, type in your name. The article  %
% class will produce a title page using the command \maketitle %
% containing the title of the document, the author name and    %
% the date.                                                    %
%                                                              %
%%%%%%%%%%%%%%%%%%%%%%%%%%%%%%%%%%%%%%%%%%%%%%%%%%%%%%%%%%%%%%%%
%
\title{MATH-UA 120 Discrete Mathematics: \\ Problem Set 1}
\author{%
    Harry Potter % Change to your name!
}
\date{Due Monday, November 11th, 2024}
%
%%%%%%%%%%%%%%%%%%%%%%%%%%%%%%%%%%%%%%%%%%%%%%%%%%%%%%%%%%%%%%%%
%                                                              %
% The body of the document is typed in between the lines       %
% \begin{document} and \end{document}.                         %
%                                                              %
%%%%%%%%%%%%%%%%%%%%%%%%%%%%%%%%%%%%%%%%%%%%%%%%%%%%%%%%%%%%%%%%
%
\begin{document}
\maketitle % this command generate the title page information that was filled in in the preamble


% The following are the asignment instructions. You should leave these alone and, after reading them, proceed to the problems.
\section*{Assignment Instructions}

\begin{itemize}
    \item These are to be written up in \LaTeX{} and turned in on Gradescope before 11:59pm of the due date.
    \item \href{https://www.google.com}{Click here} to duplicate the .tex file in Overleaf.
    \item Write your solutions inside the \texttt{solution} environment.
    \item 
    \item 
    \item 
\end{itemize}

\vfill

\section*{Statement on generative AI}

In this and other math courses, you are expected to construct clear and concise mathematical arguments based on statements proven in our text and class notes. Large language models such as ChatGPT are unable to produce this kind of solution. They also frequently generate circular logic and outright false results.
 
You may use AI to summarize content, generate study plans, create problems, or do other study-related activities. You may not ask a chatbot to solve your quiz or homework problems, or do any assessment-related activities.
 
You may use AI tools to edit your grammar and punctuation, but remember that mathematical English is not the same as academic English in other disciplines. 

\vfill

\newpage

%%%%%%%%%%%%%%%%%%%%%%%%%%%%%%%%
%%%%%       Problem 1      %%%%%
%%%%%%%%%%%%%%%%%%%%%%%%%%%%%%%%

\section*{Problem 1}

\begin{solution}
% Type your solution to Problem 1 here
\end{solution}

\newpage

%%%%%%%%%%%%%%%%%%%%%%%%%%%%%%%%
%%%%%       Problem 2      %%%%%
%%%%%%%%%%%%%%%%%%%%%%%%%%%%%%%%

\section*{Problem 2}

\begin{solution}
% Type your solution to Problem 2 here
\end{solution}

\newpage

%%%%%%%%%%%%%%%%%%%%%%%%%%%%%%%%
%%%%%       Problem 3      %%%%%
%%%%%%%%%%%%%%%%%%%%%%%%%%%%%%%%

\section*{Problem 3}

\begin{solution}
% Type your solution to Problem 3 here
\end{solution}

\newpage

%%%%%%%%%%%%%%%%%%%%%%%%%%%%%%%%
%%%%%       Problem 4      %%%%%
%%%%%%%%%%%%%%%%%%%%%%%%%%%%%%%%

\section*{Problem 4}

\begin{solution}
% Type your solution to Problem 4 here
\end{solution}

\newpage

%%%%%%%%%%%%%%%%%%%%%%%%%%%%%%%%
%%%%%       Problem 5      %%%%%
%%%%%%%%%%%%%%%%%%%%%%%%%%%%%%%%

\section*{Problem 5}

\begin{solution}
% Type your solution to Problem 5 here
\end{solution}

\newpage

%%%%%%%%%%%%%%%%%%%%%%%%%%%%%%%%
%%%%%       Problem 6      %%%%%
%%%%%%%%%%%%%%%%%%%%%%%%%%%%%%%%

\section*{Problem 6}

\begin{solution}
% Type your solution to Problem 6 here
\end{solution}

\newpage

%%%%%%%%%%%%%%%%%%%%%%%%%%%%%%%%
%%%%%       Problem 7      %%%%%
%%%%%%%%%%%%%%%%%%%%%%%%%%%%%%%%

\section*{Problem 7}

\begin{solution}
% Type your solution to Problem 7 here
\end{solution}

\newpage

%%%%%%%%%%%%%%%%%%%%%%%%%%%%%%%%
%%%%%       Problem 8      %%%%%
%%%%%%%%%%%%%%%%%%%%%%%%%%%%%%%%

\section*{Problem 8}

\begin{solution}
% Type your solution to Problem 8 here
\end{solution}

\end{document}
%
% Anything typed after \end{document} will not be included in the pdf