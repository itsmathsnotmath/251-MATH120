%%%%%%%%%%%%%%%%%%%%%%%%%%%%%%%%%%%%%%%%%%%%%%%%%%%%%%%%%%%%%%%%
%                                                              %
% This is a LaTeX file.  It is a text file that is compiled    %
% by a program called LaTeX into a pretty PDF file.            %
% If you're viewing this file on Overleaf,                     %
% you'll see that PDF in the window to the right.              %
%                                                              %
% The LaTeX macro language is complicated, so we have inserted %
% lots of documenting comments into the file.  Comments start  %
% with `%' and continue to the end of the line.                %
%                                                              %
% Comments are provided to let you, the student, understand    %
% what that part of the code is doing or to provide you with   %
% instructions.                                                % 
%                                                              %
% Skip anything else you don't understand, or ask me.          %
%                                                              %
%%%%%%%%%%%%%%%%%%%%%%%%%%%%%%%%%%%%%%%%%%%%%%%%%%%%%%%%%%%%%%%%
%
\documentclass{article}
\usepackage[left=1in,right=1in,top=1in,bottom=1in]{geometry}
% 
%%%%%%%%%%%%%%%%%%%%%%%%%%%%%%%%%%%%%%%%%%%%%%%%%%%%%%%%%%%%%%%%
%                                                              %
% This is the preamble of the document. This is where we       %
% declare the packages we need for the pdf file to compile     %
% correctly. Packages contain the different commands we        %
% need to use that allow us to format the document nicely.     %
%                                                              %
% You shouldn't need to edit any of these packages.            %
%                                                              %
%%%%%%%%%%%%%%%%%%%%%%%%%%%%%%%%%%%%%%%%%%%%%%%%%%%%%%%%%%%%%%%%
% 
\usepackage{amsmath, amsthm, amsfonts, amssymb} % These packages contain most of the commands needed to format the maths symbols.
\usepackage{enumerate} % Package that contains options for the \begin{enumerate} environment.
\usepackage{hyperref} % Package that allows hyperlinks.
\newtheorem*{theorem}{Theorem} % Defines an environment with the heading "Theorem". The * supresses the numbering.
\newtheorem*{claim}{Claim} % Defines an environment with the heading "Claim". The * supresses the numbering.
\theoremstyle{definition}
\newtheorem*{definition}{Definition} % Defines an environment with the heading "Definition". The * supresses the numbering.
\newenvironment{solution}{\bigskip\hrule{\hfill}}{\bigskip\hrule{\hfill}} % Defines an environment that draws lines to make clear where your solution starts and ends.
% 
%%%%%%%%%%%%%%%%%%%%%%%%%%%%%%%%%%%%%%%%%%%%%%%%%%%%%%%%%%%%%%%%
%                                                              %
% In the author command below, type in your name. The article  %
% class will produce a title page using the command \maketitle %
% containing the title of the document, the author name and    %
% the date.                                                    %
%                                                              %
%%%%%%%%%%%%%%%%%%%%%%%%%%%%%%%%%%%%%%%%%%%%%%%%%%%%%%%%%%%%%%%%
%
\title{\textbf{MATH-UA 120 Discrete Mathematics: \\ Problem Set 1}}
\author{%
    Mary Poppins % Change to your name!
}
\date{Due Monday, September 16th, 2024} % The due date of the assignment. All assignemets are dues at 11:59pm on the date listed.
%
%%%%%%%%%%%%%%%%%%%%%%%%%%%%%%%%%%%%%%%%%%%%%%%%%%%%%%%%%%%%%%%%
%                                                              %
% The body of the document is typed in between the lines       %
% \begin{document} and \end{document}.                         %
%                                                              %
%%%%%%%%%%%%%%%%%%%%%%%%%%%%%%%%%%%%%%%%%%%%%%%%%%%%%%%%%%%%%%%%
%
\begin{document}
\maketitle % This command generate the title page information that was filled in in the preamble.

\vfill

% The following are the asignment instructions. You should leave these alone and, after reading them, proceed to the problems.
\section*{Assignment Instructions}

\begin{itemize}
    \item These are to be written up in \LaTeX{} and turned in on Gradescope.
    \item \href{https://bit.ly/4d4g2BH}{\textbf{Click here to duplicate this \texttt{.tex} file in Overleaf}}.
    \item Write your solutions inside the \texttt{solution} environment.
    \item You are always encouraged to talk problems through with your peers and your instructor, but your write up should be done independently.
    \item Problems are graded on correctness and fluency.
    \item Some tutorials on how to use \LaTeX{} can be found \href{https://www.overleaf.com/learn/latex/Tutorials}{\underline{here}}. If you have any questions about \LaTeX{} commands you can always ask your instructor for advice.
\end{itemize}

\vfill

\section*{Statement on generative AI}

In this and other mathematics courses, you are expected to construct clear and concise mathematical arguments based on statements proven in our text and class notes. Large language models such as ChatGPT are unable to produce this kind of solution. They also frequently generate circular logic and outright false results.
 
You may use AI to summarise content, generate study plans, create problems, or do other study-related activities. You may not ask a chatbot to solve your quiz or homework problems, or do any assessment-related activities.
 
You may use AI tools to edit your grammar and punctuation, but remember that mathematical English is not the same as academic English in other disciplines. 

\vfill

\newpage

%%%%%%%%%%%%%%%%%%%%%%%%%%%%%%%%%%%%%%%%%%%%%%%%%%%%%%%%%%%%%%%%
%%%%%                       Problem 1                      %%%%%
%%%%%%%%%%%%%%%%%%%%%%%%%%%%%%%%%%%%%%%%%%%%%%%%%%%%%%%%%%%%%%%%

\section*{Problem 1}

Consider the following.

\begin{definition}
    An integer $n$ is called a \emph{perfect square} if there exists an integer $k$ such that $n=k^2$.
\end{definition}
\begin{theorem}
    There exists a non-zero perfect square that can be written as the sum of two other non-zero perfect squares.
\end{theorem}

Using only the information given above, which of the following conclusions are valid? Explain why or why not.

\begin{enumerate}[a)] % The enumerate environment produces a numbered list of items. The [a)] ensures that the items are labelled with letters instead.
    \item Let $n$ be a non-zero perfect square. There exists two non-zero perfect squares $a$ and $b$ such that $a+b=n$.

    \item If $n$ is an odd integer, then there exists perfect square $m$ such that $m+n$ is a perfect square.
\end{enumerate}

\begin{solution}

% Type your solution to Problem 1 here.

\end{solution}

%%%%%%%%%%%%%%%%%%%%%%%%%%%%%%%%%%%%%%%%%%%%%%%%%%%%%%%%%%%%%%%%

\newpage

%%%%%%%%%%%%%%%%%%%%%%%%%%%%%%%%%%%%%%%%%%%%%%%%%%%%%%%%%%%%%%%%
%%%%%                       Problem 2                      %%%%%
%%%%%%%%%%%%%%%%%%%%%%%%%%%%%%%%%%%%%%%%%%%%%%%%%%%%%%%%%%%%%%%%

\section*{Problem 2}

Without changing their meanings, write the following statements as if-then statements. Then decide if the statements are true or false. (You don't need to justify why they are true or false).

\begin{enumerate}[a)] % The enumerate environment produces a numbered list of items. The [a)] ensures that the items are labelled with letters instead.
    \item An integer is divisible by $3$ only if it is odd.

    \item Three consecutive odd integers cannot be all prime.

    \item The square of any real number is non-negative.
\end{enumerate}

\begin{solution}

% Type your solution to Problem 2 here.

\end{solution}

%%%%%%%%%%%%%%%%%%%%%%%%%%%%%%%%%%%%%%%%%%%%%%%%%%%%%%%%%%%%%%%%

\newpage

%%%%%%%%%%%%%%%%%%%%%%%%%%%%%%%%%%%%%%%%%%%%%%%%%%%%%%%%%%%%%%%%
%%%%%                       Problem 3                      %%%%%
%%%%%%%%%%%%%%%%%%%%%%%%%%%%%%%%%%%%%%%%%%%%%%%%%%%%%%%%%%%%%%%%

\section*{Problem 3}

Give a reason as to why each of the following statements are false.

\begin{enumerate}[a)] % The enumerate environment produces a numbered list of items. The [a)] ensures that the items are labelled with letters instead.
    \item There exists a positive integer $n$ such that $n^2-16$ is prime.
    
    \item For every odd prime $p$, $p^2-2$ is prime.
\end{enumerate}

\begin{solution}

% Type your solution to Problem 3 here.

\end{solution}

%%%%%%%%%%%%%%%%%%%%%%%%%%%%%%%%%%%%%%%%%%%%%%%%%%%%%%%%%%%%%%%%

\newpage

%%%%%%%%%%%%%%%%%%%%%%%%%%%%%%%%%%%%%%%%%%%%%%%%%%%%%%%%%%%%%%%%
%%%%%                       Problem 4                      %%%%%
%%%%%%%%%%%%%%%%%%%%%%%%%%%%%%%%%%%%%%%%%%%%%%%%%%%%%%%%%%%%%%%%

\section*{Problem 4}

The following claim is true, but the given proof contains a flaw, making it invalid. Find the flaw and fix the proof so that it is valid.

    \begin{claim}
        The difference between any odd integer and any even integer is odd.
    \end{claim}
    \begin{proof}
        Suppose $n$ is an odd integer and $m$ is an even integer. By the definition of odd, $n=2k+1$ where $k$ is an integer, and by the definition of even $m=2k$ where $k$ is an integer. Then $$n-m=\left(2k+1\right)-2k=1.$$ Since $1$ is odd, the claim is true.
    \end{proof}

\begin{solution}

% Type your solution to Problem 4 here.

\end{solution}

%%%%%%%%%%%%%%%%%%%%%%%%%%%%%%%%%%%%%%%%%%%%%%%%%%%%%%%%%%%%%%%%

\newpage

%%%%%%%%%%%%%%%%%%%%%%%%%%%%%%%%%%%%%%%%%%%%%%%%%%%%%%%%%%%%%%%%
%%%%%                       Problem 5                      %%%%%
%%%%%%%%%%%%%%%%%%%%%%%%%%%%%%%%%%%%%%%%%%%%%%%%%%%%%%%%%%%%%%%%

\section*{Problem 5}

The following claim and its proof are poorly written. They are both missing important information. Explain why both the claim and its proof are poorly written, then revise both the claim and proof so that any student in this course will understand it.

\begin{claim}
    If $x^2\neq0$ then $x^2>0$
\end{claim}
\begin{proof}
    If $x>0$, then $x^2=xx>0$. If $x<0$, then $-x>0$, so $\left(-x\right)\left(-x\right)>0$. i.e., $x^2>0$.
\end{proof}
\begin{solution}

% Type your solution to Problem 5 here.

\end{solution}

%%%%%%%%%%%%%%%%%%%%%%%%%%%%%%%%%%%%%%%%%%%%%%%%%%%%%%%%%%%%%%%%

\newpage

%%%%%%%%%%%%%%%%%%%%%%%%%%%%%%%%%%%%%%%%%%%%%%%%%%%%%%%%%%%%%%%%
%%%%%                       Problem 6                      %%%%%
%%%%%%%%%%%%%%%%%%%%%%%%%%%%%%%%%%%%%%%%%%%%%%%%%%%%%%%%%%%%%%%%

\section*{Problem 6}

Consider the following definitions.

\begin{definition}
    An integer is \emph{alphic} if $n=4k+1$ for some integer $k$. 
\end{definition}
\begin{definition}
    An integer is \emph{gammic} if $n=4k+3$ for some integer $k$.
\end{definition}

\begin{enumerate}[a)] % enumerate produces a numbered list of items. The [a)] ensures that the items are labelled with letters instead.
    \item Show that $-17$ is gammic.

    \item Give an example of an alphic integer and explain why it is alphic.

    \item Explain whether the following statement is true: ``If $6$ is alphic, then $8$ is gammic.''

    \item Prove that if $x$ is alphic and $y$ is gammic, then $x-y$ is even.
\end{enumerate}

\begin{solution}

% Type your solution to Problem 6 here.

\end{solution}

%%%%%%%%%%%%%%%%%%%%%%%%%%%%%%%%%%%%%%%%%%%%%%%%%%%%%%%%%%%%%%%%

\newpage

%%%%%%%%%%%%%%%%%%%%%%%%%%%%%%%%%%%%%%%%%%%%%%%%%%%%%%%%%%%%%%%%
%%%%%                       Problem 7                      %%%%%
%%%%%%%%%%%%%%%%%%%%%%%%%%%%%%%%%%%%%%%%%%%%%%%%%%%%%%%%%%%%%%%%

\section*{Problem 7}

Prove that an integer $n$ is odd if and only if $n^2-1$ is divisible by $4$.

\begin{solution}

% Type your solution to Problem 7 here.

\end{solution}

%%%%%%%%%%%%%%%%%%%%%%%%%%%%%%%%%%%%%%%%%%%%%%%%%%%%%%%%%%%%%%%%

\newpage

%%%%%%%%%%%%%%%%%%%%%%%%%%%%%%%%%%%%%%%%%%%%%%%%%%%%%%%%%%%%%%%%
%%%%%                       Problem 8                      %%%%%
%%%%%%%%%%%%%%%%%%%%%%%%%%%%%%%%%%%%%%%%%%%%%%%%%%%%%%%%%%%%%%%%

\section*{Problem 8}

Find counterexamples to each of the following statements.

\begin{enumerate}[a)] % The enumerate environment produces a numbered list of items. The [a)] ensures that the items are labelled with letters instead.
    \item For all integers $a$, $b$ and $c$, if $a\mid\left(b+c\right)$ then $a\mid b$ or $a\mid c$.

    \item For all integers $a$, $b$ and $c$, if $\left(ab\right)\mid c^2$ then $a\mid c$ and $b\mid c$.

    \item For all integers $a$ and $b$, if $a\mid b^2$ and $a\leq b$, then $a\mid b$.
\end{enumerate}

\begin{solution}

% Type your solution to Problem 8 here.

\end{solution}

%%%%%%%%%%%%%%%%%%%%%%%%%%%%%%%%%%%%%%%%%%%%%%%%%%%%%%%%%%%%%%%%

\newpage

%%%%%%%%%%%%%%%%%%%%%%%%%%%%%%%%%%%%%%%%%%%%%%%%%%%%%%%%%%%%%%%%
%%%%%                       Problem 9                      %%%%%
%%%%%%%%%%%%%%%%%%%%%%%%%%%%%%%%%%%%%%%%%%%%%%%%%%%%%%%%%%%%%%%%

\section*{Problem 9}

For each of the following claims, an attempt to disprove the statement is given. For each one, explain why the argument given is invalid. Then, give an argument that either proves or disproves the claim.

\begin{enumerate}[a)] % The enumerate environment produces a numbered list of items. The [a)] ensures that the items are labelled with letters instead.
    \item ~\vspace{-19pt}
    \begin{claim}
        If $n$ is divisible by $4$ then $3n+2$ is divisible by $6$.
    \end{claim}
    \begin{proof}
        Let $n=6$. Then $3\cdot 6+2=20$ and $20$ is not divisible by $6$. Thus the claim is false.
    \end{proof}
    
    \item ~\vspace{-19pt}
    \begin{claim}
         Every composite integer $n$ is divisible by either $2$ or $3$.
    \end{claim}
    \begin{proof}
        Let $n=1$. Then $2\not|~ 1$ and $3\not|~1$. Thus the claim is false.
    \end{proof}

    \item ~\vspace{-19pt}
    \begin{claim}
        If $n$ is a positive integer such that $n^2+4n+2$ is prime, then $n$ is odd.
    \end{claim}
    \begin{proof}
        Let $n=9$. Then $n$ is odd but $9^2+4\left(9\right)+2=119=7\cdot17$ is composite. Thus the claim is false.
    \end{proof}
\end{enumerate}

\begin{solution}

% Type your solution to Problem 9 here.

\end{solution}

%%%%%%%%%%%%%%%%%%%%%%%%%%%%%%%%%%%%%%%%%%%%%%%%%%%%%%%%%%%%%%%%

\end{document}
%
% Anything typed after \end{document} will not be included in the pdf