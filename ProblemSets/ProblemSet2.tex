%%%%%%%%%%%%%%%%%%%%%%%%%%%%%%%%%%%%%%%%%%%%%%%%%%%%%%%%%%%%%%%%
%                                                              %
% This is a LaTeX file.  It is a text file that is compiled    %
% by a program called LaTeX into a pretty PDF file.            %
% If you're viewing this file on Overleaf,                     %
% you'll see that PDF in the window to the right.              %
%                                                              %
% The LaTeX macro language is complicated, so we have inserted %
% lots of documenting comments into the file.  Comments start  %
% with `%' and continue to the end of the line.                %
%                                                              %
% Comments are provided to let you, the student, understand    %
% what that part of the code is doing or to provide you with   %
% instructions.                                                % 
%                                                              %
% Skip anything else you don't understand, or ask me.          %
%                                                              %
%%%%%%%%%%%%%%%%%%%%%%%%%%%%%%%%%%%%%%%%%%%%%%%%%%%%%%%%%%%%%%%%
%
\documentclass{article}
\usepackage[left=1in,right=1in,top=1in,bottom=1in]{geometry}
% 
%%%%%%%%%%%%%%%%%%%%%%%%%%%%%%%%%%%%%%%%%%%%%%%%%%%%%%%%%%%%%%%%
%                                                              %
% This is the preamble of the document. This is where we       %
% declare the packages we need for the pdf file to compile     %
% correctly. Packages contain the different commands we        %
% need to use that allow us to format the document nicely.     %
%                                                              %
% You shouldn't need to edit any of these packages.            %
%                                                              %
%%%%%%%%%%%%%%%%%%%%%%%%%%%%%%%%%%%%%%%%%%%%%%%%%%%%%%%%%%%%%%%%
% 
\usepackage{amsmath, amsthm, amsfonts, amssymb} % These packages contain most of the commands needed to format the maths symbols.
\usepackage{enumerate} % Package that contains options for the \begin{enumerate} environment.
\usepackage{hyperref} % Package that allows hyperlinks.
\newtheorem*{theorem}{Theorem} % Defines an environment with the heading "Theorem". The * supresses the numbering.
\newtheorem*{claim}{Claim} % Defines an environment with the heading "Claim". The * supresses the numbering.
\theoremstyle{definition}
\newtheorem*{definition}{Definition} % Defines an environment with the heading "Definition". The * supresses the numbering.
\newenvironment{solution}{\bigskip\hrule{\hfill}}{\bigskip\hrule{\hfill}} % Defines an environment that draws lines to make clear where your solution starts and ends.
% 
%%%%%%%%%%%%%%%%%%%%%%%%%%%%%%%%%%%%%%%%%%%%%%%%%%%%%%%%%%%%%%%%
%                                                              %
% In the author command below, type in your name. The article  %
% class will produce a title page using the command \maketitle %
% containing the title of the document, the author name and    %
% the date.                                                    %
%                                                              %
%%%%%%%%%%%%%%%%%%%%%%%%%%%%%%%%%%%%%%%%%%%%%%%%%%%%%%%%%%%%%%%%
%
\title{\textbf{MATH-UA 120 Discrete Mathematics: \\ Problem Set 2}}
\author{%
    James Bond % Change to your name!
}
\date{Due Monday, September 23rd, 2024} % The due date of the assignment. All assignemets are dues at 11:59pm on the date listed
%
%%%%%%%%%%%%%%%%%%%%%%%%%%%%%%%%%%%%%%%%%%%%%%%%%%%%%%%%%%%%%%%%
%                                                              %
% The body of the document is typed in between the lines       %
% \begin{document} and \end{document}.                         %
%                                                              %
%%%%%%%%%%%%%%%%%%%%%%%%%%%%%%%%%%%%%%%%%%%%%%%%%%%%%%%%%%%%%%%%
%
\begin{document}
\maketitle % This command generate the title page information that was filled in in the preamble

\vfill

% The following are the asignment instructions. You should leave these alone and, after reading them, proceed to the problems.
\section*{Assignment Instructions}

\begin{itemize}
    \item These are to be written up in \LaTeX{} and turned in on Gradescope.
    \item \href{https://bit.ly/3XfKMtA}{\textbf{Click here to duplicate this \texttt{.tex} file in Overleaf}}.
    \item Write your solutions inside the \texttt{solution} environment.
    \item You are always encouraged to talk problems through with your peers and your instructor, but your write up should be done independently.
    \item Problems are graded on correctness and fluency.
    \item Some tutorials on how to use \LaTeX{} can be found \href{https://www.overleaf.com/learn/latex/Tutorials}{\underline{here}}. If you have any questions about \LaTeX{} commands you can always ask your instructor for advice.
\end{itemize}

\vfill

\section*{Statement on generative AI}

In this and other mathematics courses, you are expected to construct clear and concise mathematical arguments based on statements proven in our text and class notes. Large language models such as ChatGPT are unable to produce this kind of solution. They also frequently generate circular logic and outright false results.
 
You may use AI to summarise content, generate study plans, create problems, or do other study-related activities. You may not ask a chatbot to solve your quiz or homework problems, or do any assessment-related activities.
 
You may use AI tools to edit your grammar and punctuation, but remember that mathematical English is not the same as academic English in other disciplines. 

\vfill

\newpage

%%%%%%%%%%%%%%%%%%%%%%%%%%%%%%%%%%%%%%%%%%%%%%%%%%%%%%%%%%%%%%%%
%%%%%                       Problem 1                      %%%%%
%%%%%%%%%%%%%%%%%%%%%%%%%%%%%%%%%%%%%%%%%%%%%%%%%%%%%%%%%%%%%%%%

\section*{Problem 1}

Disprove each of the following statements.

\begin{enumerate}[a)] % The enumerate environment produces a numbered list of items. The [a)] ensures that the items are labelled with letters instead.
        \item Every triangle has at least one obtuse angle.
        \item For all integers $x, y$, if $x$ divides $y^2$, then $x$ divides $y$.
        \item For every positive non-prime integer $n$, if some prime $p$ divides $n$, 
            then some other prime $q$ ($q\neq p$) also divides $n$.
        \item For all integers $n$, if $n^5-n$ is even, then $n$ is even.
        \item For all natural numbers $n$, the integer $n^2+17n+17$ is prime.
    \end{enumerate}

\begin{solution}

% Type your solution to Problem 1 here.

\end{solution}

%%%%%%%%%%%%%%%%%%%%%%%%%%%%%%%%%%%%%%%%%%%%%%%%%%%%%%%%%%%%%%%%

\newpage

%%%%%%%%%%%%%%%%%%%%%%%%%%%%%%%%%%%%%%%%%%%%%%%%%%%%%%%%%%%%%%%%
%%%%%                       Problem 2                      %%%%%
%%%%%%%%%%%%%%%%%%%%%%%%%%%%%%%%%%%%%%%%%%%%%%%%%%%%%%%%%%%%%%%%

\section*{Problem 2}

Prove or disprove the following statement: If $n$ is odd, then $8\mid\left(n^4+4n^2+11\right)$.

\begin{solution}

% Type your solution to Problem 2 here.

\end{solution}

%%%%%%%%%%%%%%%%%%%%%%%%%%%%%%%%%%%%%%%%%%%%%%%%%%%%%%%%%%%%%%%%

\newpage

%%%%%%%%%%%%%%%%%%%%%%%%%%%%%%%%%%%%%%%%%%%%%%%%%%%%%%%%%%%%%%%%
%%%%%                       Problem 3                      %%%%%
%%%%%%%%%%%%%%%%%%%%%%%%%%%%%%%%%%%%%%%%%%%%%%%%%%%%%%%%%%%%%%%%

\section*{Problem 3}

Prove or disprove the following statement: Let $a$, $b$ and $c$ be integers. If $a\mid c$ and $b\mid c$ then $\left(a+b\right)\mid c$.

% \mid produces a vertical bar, |. The alternatives, \vert or, simply, |, produce the same symbol, but the spacing is less nice. So \mid is preferred.

\begin{solution}

% Type your solution to Problem 3 here.

\end{solution}

%%%%%%%%%%%%%%%%%%%%%%%%%%%%%%%%%%%%%%%%%%%%%%%%%%%%%%%%%%%%%%%%

\newpage

%%%%%%%%%%%%%%%%%%%%%%%%%%%%%%%%%%%%%%%%%%%%%%%%%%%%%%%%%%%%%%%%
%%%%%                       Problem 4                      %%%%%
%%%%%%%%%%%%%%%%%%%%%%%%%%%%%%%%%%%%%%%%%%%%%%%%%%%%%%%%%%%%%%%%

\section*{Problem 4}

Another method to prove that certain Boolean formulas are tautologies is to use the properties in Theorem~7.2 together with the fact that $x \rightarrow y$ is equivalent to $(\lnot x) \lor y$ (Proposition~7.3) For example, Exercise 7.11, part (b) asks you to establish that the formula $(x \land (x \rightarrow y)) \rightarrow y$ is  a tautology.  Here is a derivation of that fact:
    % This is an "align" environment for aligning mathematical equations. The asterisk, *, suppresses numbering on each line. 
    % Use ampersands, &, to set the alignment.
    % Using more ampersands, &, in the same line creates more "columns" of alignment.
    % The first one means there will be alignment around the equal sign.
    % The double-ampersands put a second column, left-justified. This is useful for clarifying text you might choose to include
    % Use double-backslashes, \\, to start a new line.
    \begin{align*}
        (x \land (x \rightarrow y)) \rightarrow y
            & = [x \land (\lnot x \lor y)] \rightarrow y
            && \text{translate $\rightarrow$} 
                \\
            & = [(x \land \lnot x) \lor (x \land y)] \rightarrow y
            && \text{distributive}
                \\
            & = [\mathrm{FALSE} \lor (x\land y)] \rightarrow y
            && \text{inverse elements}
                \\
            & = (x\land y) \rightarrow y
            && \text{identity element}
                \\
            & = \lnot(x\land y) \lor y
            && \text{translate $\rightarrow$}
                \\
            & = (\lnot x \lor \lnot y) \lor y
            && \text{De~Morgan's laws}
                \\
            & = \lnot x \lor (\lnot y \lor y)
            && \text{associativity}
                \\
            & = \lnot x \lor \mathrm{TRUE}
            && \text{inverse elements}
                \\
            & = \mathrm{TRUE}
            && \text{identity element}
                \\
    \end{align*}
    Use this technique [not truth tables] to prove that these formulas are tautologies:
    \begin{enumerate}[a)] % The enumerate environment produces a numbered list of items. The [a)] ensures that the items are labelled with letters instead.
        \item $(\lnot x \land (x \lor y)) \rightarrow y$
        \item $\lnot ( (x \rightarrow y) \rightarrow  y) \rightarrow \lnot x$
    \end{enumerate}

    % The commands \lnot, \land and \lor create the symbols for the logical not, and and or, respectively. 
    % As an alternative, you can use \neg, \wedge and \vee to create these symbols.

\begin{solution}

% Type your solution to Problem 4 here.

\end{solution}

%%%%%%%%%%%%%%%%%%%%%%%%%%%%%%%%%%%%%%%%%%%%%%%%%%%%%%%%%%%%%%%%

\newpage

%%%%%%%%%%%%%%%%%%%%%%%%%%%%%%%%%%%%%%%%%%%%%%%%%%%%%%%%%%%%%%%%
%%%%%                       Problem 5                      %%%%%
%%%%%%%%%%%%%%%%%%%%%%%%%%%%%%%%%%%%%%%%%%%%%%%%%%%%%%%%%%%%%%%%

\section*{Problem 5}

Let the following statements be given.
    \begin{align*}
        x & = \text{``There is rain.''}\\
	y & = \text{``There is traffic.''}\\
        z & = \text{``The football game is happening.''}\\
	w & = \text{``The trophy was awarded.''}
    \end{align*}
    \begin{enumerate}[a)] % The enumerate environment produces a numbered list of items. The [a)] ensures that the items are labelled with letters instead.
	\item Rewrite the following statements as Boolean expressions.
        \begin{enumerate}[i.] % The enumerate environment produces a numbered list of items. The [a)] ensures that the items are labelled with roman numerals instead.
            \item If there is no rain or no traffic, then the football game is happening.
		\item If the football game is happening, then a trophy will be awarded.
	\end{enumerate}
	\item Construct a truth table for the statements in part (a).
	\item Suppose that the statements given in part (a) are true, and suppose also that the trophy was not awarded. Did it rain? Justify your answer using the truth table.
    \end{enumerate}

\begin{solution}

% Type your solution to Problem 5 here.

% The tabular environment provides a convenient way to generate tables. You can also the website https://www.tablesgenerator.com/ to generate code for a table using a graphical interface.
% You can uncomment the following to start your table off.

% \begin{center}
%     \begin{tabular}{| c | c | c | c || c | c | c |} % The bars | indicate that a border should be drawn in that column. The letters l, c and r are used to specify the alignment of the text in the cell. (left, centre or right, respectively).
%         \hline % \hline draws a line across the bottom of the row of the table. Putting it before any table content draws a line across the top of the table
%         $x$ & $y$ & $z$ & $w$ & $\neg x \vee \neg y$ & $(\neg x \vee \neg y) \to z$ & $z \to w$ \\\hline % Use ampersands, &, to separate the cells
%         T & T & T & T &  &  & \\
%         \hline % \hline draws a line across the bottom of the row of the table. Putting it after any table content draws a line across the bottom of the table
%     \end{tabular}
% \end{center}

\end{solution}

%%%%%%%%%%%%%%%%%%%%%%%%%%%%%%%%%%%%%%%%%%%%%%%%%%%%%%%%%%%%%%%%

\newpage

%%%%%%%%%%%%%%%%%%%%%%%%%%%%%%%%%%%%%%%%%%%%%%%%%%%%%%%%%%%%%%%%
%%%%%                       Problem 6                      %%%%%
%%%%%%%%%%%%%%%%%%%%%%%%%%%%%%%%%%%%%%%%%%%%%%%%%%%%%%%%%%%%%%%%

\section*{Problem 6}

Suppose we want to make a list of length 5 from the letters $A$, $B$, $C$, $D$, $E$, $F$, $G$, $H$, $I$ and  $J$. Provide a brief reasoning for the following questions.
    \begin{enumerate}[a)] % The enumerate environment produces a numbered list of items. The [a)] ensures that the items are labelled with letters instead.
        \item How many such lists can be made if repetition is not allowed and the list must begin with a vowel?
        \item How many such lists can be made if repetition is not allowed and the list must contain exactly one $A$?
        \item How many such lists can be made if repetition is not allowed and the list must begin with a vowel and end with a vowel?
        \item How many such lists can be made if repetition is not allowed, and the list must begin with a vowel or end with a vowel?
        \item How many such lists can be made if repetition is not allowed and the list must contain exactly two vowels?
    \end{enumerate}

\begin{solution}

% Type your solution to Problem 6 here.

\end{solution}

%%%%%%%%%%%%%%%%%%%%%%%%%%%%%%%%%%%%%%%%%%%%%%%%%%%%%%%%%%%%%%%%

\newpage

%%%%%%%%%%%%%%%%%%%%%%%%%%%%%%%%%%%%%%%%%%%%%%%%%%%%%%%%%%%%%%%%
%%%%%                       Problem 7                      %%%%%
%%%%%%%%%%%%%%%%%%%%%%%%%%%%%%%%%%%%%%%%%%%%%%%%%%%%%%%%%%%%%%%%

\section*{Problem 7}

An art collector wishes to display their nine paintings on a wall along a long corridor. They have three paintings whose colour scheme is predominantly red, four paintings whose colour scheme is predominantly blue and two paintings whose colour scheme is predominantly green.
    \begin{enumerate}[a)] % The enumerate environment produces a numbered list of items. The [a)] ensures that the items are labelled with letters instead.
        \item In how many different ways can these paintings be arranged?
        \item In how many different ways can these paintings be arranged if the paintings of the same colour scheme should be displayed next to each other?
        \item In how many different ways can these paintings be arranged if the green paintings cannot be next to each other?
    \end{enumerate}

\begin{solution}

% Type your solution to Problem 7 here.

\end{solution}

%%%%%%%%%%%%%%%%%%%%%%%%%%%%%%%%%%%%%%%%%%%%%%%%%%%%%%%%%%%%%%%%

\end{document}
%
% Anything typed after \end{document} will not be included in the pdf