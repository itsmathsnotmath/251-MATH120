%%%%%%%%%%%%%%%%%%%%%%%%%%%%%%%%%%%%%%%%%%%%%%%%%%%%%%%%%%%%%%%%
%                                                              %
% This is a LaTeX file.  It is a text file that is compiled    %
% by a program called LaTeX into a pretty PDF file.            %
% If you're viewing this file on Overleaf,                     %
% you'll see that PDF in the window to the right.              %
%                                                              %
% The LaTeX macro language is complicated, so we have inserted %
% lots of documenting comments into the file.  Comments start  %
% with `%' and continue to the end of the line.                %
%                                                              %
% Comments are provided to let you, the student, understand    %
% what that part of the code is doing or to provide you with   %
% instructions.                                                % 
%                                                              %
% Skip anything else you don't understand, or ask me.          %
%                                                              %
%%%%%%%%%%%%%%%%%%%%%%%%%%%%%%%%%%%%%%%%%%%%%%%%%%%%%%%%%%%%%%%%
%
\documentclass{article}
\usepackage[left=1in,right=1in,top=1in,bottom=1in]{geometry}
% 
%%%%%%%%%%%%%%%%%%%%%%%%%%%%%%%%%%%%%%%%%%%%%%%%%%%%%%%%%%%%%%%%
%                                                              %
% This is the preamble of the document. This is where we       %
% declare the packages we need for the pdf file to compile     %
% correctly. Packages contain the different commands we        %
% need to use that allow us to format the document nicely.     %
%                                                              %
% You shouldn't need to edit any of these packages.            %
%                                                              %
%%%%%%%%%%%%%%%%%%%%%%%%%%%%%%%%%%%%%%%%%%%%%%%%%%%%%%%%%%%%%%%%
% 
\usepackage{amsmath, amsthm, amsfonts} % These packages contain most of the commands needed to format the maths symbols.
\usepackage{enumerate} % Package that contains options for the \begin{enumerate} environment.
\usepackage{hyperref} % Package that allows hyperlinks.
\newtheorem*{theorem}{Theorem} % Defines an environment with the heading "Theorem". The * supresses the numbering.
\newtheorem*{claim}{Claim} % Defines an environment with the heading "Claim". The * supresses the numbering.
\theoremstyle{definition}
\newtheorem*{definition}{Definition} % Defines an environment with the heading "Definition". The * supresses the numbering.
\newtheorem*{proposition}{Proposition} % Defines an environment with the heading "Proposition". The * supresses the numbering.
\newenvironment{solution}{\bigskip\hrule{\hfill}}{\bigskip\hrule{\hfill}} % Defines an environment that draws lines to make clear where your solution starts and ends.
% 
%%%%%%%%%%%%%%%%%%%%%%%%%%%%%%%%%%%%%%%%%%%%%%%%%%%%%%%%%%%%%%%%
%                                                              %
% In the author command below, type in your name. The article  %
% class will produce a title page using the command \maketitle %
% containing the title of the document, the author name and    %
% the date.                                                    %
%                                                              %
%%%%%%%%%%%%%%%%%%%%%%%%%%%%%%%%%%%%%%%%%%%%%%%%%%%%%%%%%%%%%%%%
%
\title{\textbf{MATH-UA 120 Discrete Mathematics: \\ Problem Set 4}}
\author{%
    Sherlock Holmes % Change to your name!
}
\date{Due Monday, October 21st, 2024} % The due date of the assignment. All assignemets are dues at 11:59pm on the date listed
%
%%%%%%%%%%%%%%%%%%%%%%%%%%%%%%%%%%%%%%%%%%%%%%%%%%%%%%%%%%%%%%%%
%                                                              %
% The body of the document is typed in between the lines       %
% \begin{document} and \end{document}.                         %
%                                                              %
%%%%%%%%%%%%%%%%%%%%%%%%%%%%%%%%%%%%%%%%%%%%%%%%%%%%%%%%%%%%%%%%
%
\begin{document}
\maketitle % lThis command generate the title page information that was filled in in the preamble

\vfill

% The following are the asignment instructions. You should leave these alone and, after reading them, proceed to the problems.
\section*{Assignment Instructions}

\begin{itemize}
    \item These are to be written up in \LaTeX{} and turned in on Gradescope.
    \item \href{https://bit.ly/3ALpQD3}{\textbf{Click here to duplicate this \texttt{.tex} file in Overleaf}}.
    \item Write your solutions inside the \texttt{solution} environment.
    \item You are always encouraged to talk problems through with your peers and your instructor, but your write up should be done independently.
    \item Problems are graded on correctness and fluency.
    \item Unless stated otherwise, all calculations require justification.
    \item Some tutorials on how to use \LaTeX{} can be found \href{https://www.overleaf.com/learn/latex/Tutorials}{\underline{here}}. If you have any questions about \LaTeX{} commands you can always ask your instructor for advice.
\end{itemize}

\vfill

\section*{Statement on generative AI}

In this and other mathematics courses, you are expected to construct clear and concise mathematical arguments based on statements proven in our text and class notes. Large language models such as ChatGPT are unable to produce this kind of solution. They also frequently generate circular logic and outright false results.
 
You may use AI to summarise content, generate study plans, create problems, or do other study-related activities. You may not ask a chatbot to solve your quiz or homework problems, or do any assessment-related activities.
 
You may use AI tools to edit your grammar and punctuation, but remember that mathematical English is not the same as academic English in other disciplines. 

\vfill

\newpage

%%%%%%%%%%%%%%%%%%%%%%%%%%%%%%%%%%%%%%%%%%%%%%%%%%%%%%%%%%%%%%%%
%%%%%                       Problem 1                      %%%%%
%%%%%%%%%%%%%%%%%%%%%%%%%%%%%%%%%%%%%%%%%%%%%%%%%%%%%%%%%%%%%%%%

\section*{Problem 1}
Define a relation $R$ on $\mathbb{N}\times\mathbb{N}$ by $$\left(a,b\right)\mathrel{R}\left(c,d\right)\Longleftrightarrow3a+d=b+3c.$$
\begin{enumerate}[a)] % The enumerate environment produces a numbered list of items. The [a)] ensures that the items are labelled with letters instead.
    \item Show that $R$ is an equivalence relation,
    \item Describe, as simply as possible, the equivalence classes $\left[\left(0,0\right)\right]$, $\left[\left(1,2\right)\right]$ and $\left[\left(2,1\right)\right]$.
\end{enumerate}
\begin{solution}

% Type your solution to Problem 1 here.

\end{solution}

%%%%%%%%%%%%%%%%%%%%%%%%%%%%%%%%%%%%%%%%%%%%%%%%%%%%%%%%%%%%%%%%

\newpage

%%%%%%%%%%%%%%%%%%%%%%%%%%%%%%%%%%%%%%%%%%%%%%%%%%%%%%%%%%%%%%%%
%%%%%                       Problem 2                      %%%%%
%%%%%%%%%%%%%%%%%%%%%%%%%%%%%%%%%%%%%%%%%%%%%%%%%%%%%%%%%%%%%%%%

\section*{Problem 2}
Determine if the given relations have the following properties:
\begin{itemize}
    \item irreflexive,
    \item antisymmetric,
    \item transitive.
\end{itemize}
Either prove they have the property or provide a counterexample.
\begin{enumerate}[a)] % The enumerate environment produces a numbered list of items. The [a)] ensures that the items are labelled with letters instead.
    \item For $x,y\in\mathbb{Z}$, $x\mathrel{R}y\Longleftrightarrow\left|x-y\right|>0$.
    \item For $x,y\in\mathbb{Z}$, $x\mathrel{R}y$ means that $x$ and $y$ have a common prime factor, that is, a prime number that divides both $x$ and $y$.
    \item For $x,y\in 2^{\mathbb{Z}}$, $x\mathrel{R}y\Longleftrightarrow x\cap y=\emptyset$.
\end{enumerate}
\begin{solution}

% Type your solution to Problem 2 here.

\end{solution}

%%%%%%%%%%%%%%%%%%%%%%%%%%%%%%%%%%%%%%%%%%%%%%%%%%%%%%%%%%%%%%%%

\newpage

%%%%%%%%%%%%%%%%%%%%%%%%%%%%%%%%%%%%%%%%%%%%%%%%%%%%%%%%%%%%%%%%
%%%%%                       Problem 3                      %%%%%
%%%%%%%%%%%%%%%%%%%%%%%%%%%%%%%%%%%%%%%%%%%%%%%%%%%%%%%%%%%%%%%%

\section*{Problem 3}
Determine all of the (distinct) equivalence classes of the following equivalence relations. Provide a brief justification on how you found all of the equivalence classes for each relation. \medskip

\noindent\emph{There is no need to prove that they are equivalence relations; but you are encourage to prove they are equivalence relations in your own practice.}

\begin{enumerate}[a)] % The enumerate environment produces a numbered list of items. The [a)] ensures that the items are labelled with letters instead.
    \item Consider the set $A=\{0,1,2,\dots,8\}$. Define a relation $R$ on $A$ by $a\mathrel{R}b\Longleftrightarrow a^2\equiv b^2\pmod{9}$.
    \item Consider $\mathbb{Z}\times\mathbb{Z}$. Define a relation on $R$ by $\left(a,b\right)\mathrel{R}\left(c,d\right)\Longleftrightarrow a+b=c+d$.
\end{enumerate}
\begin{solution}

% Type your solution to Problem 3 here.

\end{solution}

%%%%%%%%%%%%%%%%%%%%%%%%%%%%%%%%%%%%%%%%%%%%%%%%%%%%%%%%%%%%%%%%

\newpage

%%%%%%%%%%%%%%%%%%%%%%%%%%%%%%%%%%%%%%%%%%%%%%%%%%%%%%%%%%%%%%%%
%%%%%                       Problem 4                      %%%%%
%%%%%%%%%%%%%%%%%%%%%%%%%%%%%%%%%%%%%%%%%%%%%%%%%%%%%%%%%%%%%%%%

\section*{Problem 4}
Let $R$ be a relation on a set $A$. Prove $R\cup R^{-1}$ is symmetric.
\begin{solution}

% Type your solution to Problem 4 here.

\end{solution}

%%%%%%%%%%%%%%%%%%%%%%%%%%%%%%%%%%%%%%%%%%%%%%%%%%%%%%%%%%%%%%%%

\newpage

%%%%%%%%%%%%%%%%%%%%%%%%%%%%%%%%%%%%%%%%%%%%%%%%%%%%%%%%%%%%%%%%
%%%%%                       Problem 5                      %%%%%
%%%%%%%%%%%%%%%%%%%%%%%%%%%%%%%%%%%%%%%%%%%%%%%%%%%%%%%%%%%%%%%%

\section*{Problem 5}
\begin{enumerate}[a)] % The enumerate environment produces a numbered list of items. The [a)] ensures that the items are labelled with letters instead.
    \item Let $X$ be a set with $n$ elements. How many possible reflexive relations on $X$ are there?
    \item Let $A$ be a set of size $n$. For $x,y\in 2^A$, $x\mathrel{R}y\Longleftrightarrow x\cap y=\emptyset$. What is the cardinality of $R$?
    \item What is the total number of partitions of $\left\{1,2,\dots,100\right\}$ that have exactly two parts in the partition? Remember, both parts should be non-empty.
    \item Suppose that a single character is stored in a computer using eight bits. How many bit patterns have at least two $1$'s? (Bits can either be $0$ or $1$).
    \item Twenty people are to be divided into two teams with ten players on each team. In how many ways can this be done?
    \item Thirty-five discrete maths students are to be divided into seven discussion groups, each consisting of five students. In how many ways can this be done?
\end{enumerate}
\begin{solution}

% Type your solution to Problem 5 here.

\end{solution}

%%%%%%%%%%%%%%%%%%%%%%%%%%%%%%%%%%%%%%%%%%%%%%%%%%%%%%%%%%%%%%%%

\newpage

%%%%%%%%%%%%%%%%%%%%%%%%%%%%%%%%%%%%%%%%%%%%%%%%%%%%%%%%%%%%%%%%
%%%%%                       Problem 6                      %%%%%
%%%%%%%%%%%%%%%%%%%%%%%%%%%%%%%%%%%%%%%%%%%%%%%%%%%%%%%%%%%%%%%%

\section*{Problem 6}
We are to place five posters along the wall of a corridor. There are six movie posters, five music posters and eight maths posters to choose from. In each of the following cases, determine the number of arrangements and give a brief justification in words.
\begin{enumerate}[a)] % The enumerate environment produces a numbered list of items. The [a)] ensures that the items are labelled with letters instead.
    \item How many arrangements of posters are there if there are no restrictions?
    \item How many arrangements of posters are there if we cannot have more than two maths posters?
    \item How many arrangements of posters are there if we cannot have two music posters side-by-side?
\end{enumerate}
\emph{You can (and should) leave your answers in terms of factorials and binomial coefficients where appropriate.}
\begin{solution}

% Type your solution to Problem 6 here.

\end{solution}

%%%%%%%%%%%%%%%%%%%%%%%%%%%%%%%%%%%%%%%%%%%%%%%%%%%%%%%%%%%%%%%%

\newpage

%%%%%%%%%%%%%%%%%%%%%%%%%%%%%%%%%%%%%%%%%%%%%%%%%%%%%%%%%%%%%%%%
%%%%%                       Problem 7                      %%%%%
%%%%%%%%%%%%%%%%%%%%%%%%%%%%%%%%%%%%%%%%%%%%%%%%%%%%%%%%%%%%%%%%

\section*{Problem 7}
Let $n$ be a nonnegative integer. Give a combinatorial proof of the identity $$3^n=\binom{n}{n}\cdot2^n+\binom{n}{n-1}\cdot2^{n-1}+\binom{n}{n-2}\cdot2^{n-2}+\cdots+\binom{n}{0}\cdot2^0.$$ You may not manipulate the question algebraically. Only a combinatorial proof will be accepted.
\begin{solution}

% Type your solution to Problem 7 here.

\end{solution}

%%%%%%%%%%%%%%%%%%%%%%%%%%%%%%%%%%%%%%%%%%%%%%%%%%%%%%%%%%%%%%%%

\newpage

%%%%%%%%%%%%%%%%%%%%%%%%%%%%%%%%%%%%%%%%%%%%%%%%%%%%%%%%%%%%%%%%
%%%%%                       Problem 8                      %%%%%
%%%%%%%%%%%%%%%%%%%%%%%%%%%%%%%%%%%%%%%%%%%%%%%%%%%%%%%%%%%%%%%%

\section*{Problem 8}
Let $n$ be a positive integer. Give a combinatorial proof of the identity $$1\cdot n+2\cdot\left(n-1\right)+3\cdot\left(n-2\right)+\cdots+\left(n-1\right)\cdot2+n\cdot1=\binom{n+2}{3}.$$ You may not manipulate the question algebraically. Only a combinatorial proof will be accepted.
\begin{solution}

% Type your solution to Problem 8 here.

\end{solution}

%%%%%%%%%%%%%%%%%%%%%%%%%%%%%%%%%%%%%%%%%%%%%%%%%%%%%%%%%%%%%%%%

\end{document}
%
% Anything typed after \end{document} will not be included in the pdf