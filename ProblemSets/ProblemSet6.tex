%%%%%%%%%%%%%%%%%%%%%%%%%%%%%%%%%%%%%%%%%%%%%%%%%%%%%%%%%%%%%%%%
%                                                              %
% This is a LaTeX file.  It is a text file that is compiled    %
% by a program called LaTeX into a pretty PDF file.            %
% If you're viewing this file on Overleaf,                     %
% you'll see that PDF in the window to the right.              %
%                                                              %
% The LaTeX macro language is complicated, so we have inserted %
% lots of documenting comments into the file.  Comments start  %
% with `%' and continue to the end of the line.                %
%                                                              %
% Comments are provided to let you, the student, understand    %
% what that part of the code is doing or to provide you with   %
% instructions.                                                % 
%                                                              %
% Skip anything else you don't understand, or ask me.          %
%                                                              %
%%%%%%%%%%%%%%%%%%%%%%%%%%%%%%%%%%%%%%%%%%%%%%%%%%%%%%%%%%%%%%%%
%
\documentclass{article}
\usepackage[left=1in,right=1in,top=1in,bottom=1in]{geometry}
% 
%%%%%%%%%%%%%%%%%%%%%%%%%%%%%%%%%%%%%%%%%%%%%%%%%%%%%%%%%%%%%%%%
%                                                              %
% This is the preamble of the document. This is where we       %
% declare the packages we need for the pdf file to compile     %
% correctly. Packages contain the different commands we        %
% need to use that allow us to format the document nicely.     %
%                                                              %
% You shouldn't need to edit any of these packages.            %
%                                                              %
%%%%%%%%%%%%%%%%%%%%%%%%%%%%%%%%%%%%%%%%%%%%%%%%%%%%%%%%%%%%%%%%
% 
\usepackage{amsmath, amsthm, amsfonts} % These packages contain most of the commands needed to format the maths symbols.
\usepackage{enumerate} % Package that contains options for the \begin{enumerate} environment.
\usepackage{hyperref} % Package that allows hyperlinks.
\newtheorem*{theorem}{Theorem} % Defines an environment with the heading "Theorem". The * supresses the numbering.
\newtheorem*{claim}{Claim} % Defines an environment with the heading "Claim". The * supresses the numbering.
\theoremstyle{definition}
\newtheorem*{definition}{Definition} % Defines an environment with the heading "Definition". The * supresses the numbering.
\newenvironment{solution}{\bigskip\hrule{\hfill}}{\bigskip\hrule{\hfill}} % Defines an environment that draws lines to make clear where your solution starts and ends.
% 
%%%%%%%%%%%%%%%%%%%%%%%%%%%%%%%%%%%%%%%%%%%%%%%%%%%%%%%%%%%%%%%%
%                                                              %
% In the author command below, type in your name. The article  %
% class will produce a title page using the command \maketitle %
% containing the title of the document, the author name and    %
% the date.                                                    %
%                                                              %
%%%%%%%%%%%%%%%%%%%%%%%%%%%%%%%%%%%%%%%%%%%%%%%%%%%%%%%%%%%%%%%%
%
\title{\textbf{MATH-UA 120 Discrete Mathematics: \\ Problem Set 6}}
\author{%
    Alfred Pennyworth % Change to your name!
}
\date{Due Monday, November 11th, 2024} % The due date of the assignment. All assignemets are dues at 11:59pm on the date listed
%
%%%%%%%%%%%%%%%%%%%%%%%%%%%%%%%%%%%%%%%%%%%%%%%%%%%%%%%%%%%%%%%%
%                                                              %
% The body of the document is typed in between the lines       %
% \begin{document} and \end{document}.                         %
%                                                              %
%%%%%%%%%%%%%%%%%%%%%%%%%%%%%%%%%%%%%%%%%%%%%%%%%%%%%%%%%%%%%%%%
%
\begin{document}
\maketitle % This command generate the title page information that was filled in in the preamble

\vfill

% The following are the asignment instructions. You should leave these alone and, after reading them, proceed to the problems.
\section*{Assignment Instructions}

\begin{itemize}
    \item These are to be written up in \LaTeX{} and turned in on Gradescope.
    \item \href{https://bit.ly/3AWJyvt}{\textbf{Click here to duplicate this \texttt{.tex} file in Overleaf}}.
    \item Write your solutions inside the \texttt{solution} environment.
    \item You are always encouraged to talk problems through with your peers and your instructor, but your write up should be done independently.
    \item Problems are graded on correctness and fluency.
    \item Unless stated otherwise, all answers require justification.
    \item Some tutorials on how to use \LaTeX{} can be found \href{https://www.overleaf.com/learn/latex/Tutorials}{\underline{here}}. If you have any questions about \LaTeX{} commands you can always ask your instructor for advice.
\end{itemize}

\vfill

\section*{Statement on generative AI}

In this and other mathematics courses, you are expected to construct clear and concise mathematical arguments based on statements proven in our text and class notes. Large language models such as ChatGPT are unable to produce this kind of solution. They also frequently generate circular logic and outright false results.
 
You may use AI to summarise content, generate study plans, create problems, or do other study-related activities. You may not ask a chatbot to solve your quiz or homework problems, or do any assessment-related activities.
 
You may use AI tools to edit your grammar and punctuation, but remember that mathematical English is not the same as academic English in other disciplines. 

\vfill

\newpage

%%%%%%%%%%%%%%%%%%%%%%%%%%%%%%%%%%%%%%%%%%%%%%%%%%%%%%%%%%%%%%%%
%%%%%                       Problem 1                      %%%%%
%%%%%%%%%%%%%%%%%%%%%%%%%%%%%%%%%%%%%%%%%%%%%%%%%%%%%%%%%%%%%%%%

\section*{Problem 1}
For each of the following functions, determine whether it is one-to-one and/or onto. 
\begin{enumerate}[a)] % The enumerate environment produces a numbered list of items. The [a)] ensures that the items are labelled with letters instead.
    \item $f:\mathbb{Z}\longrightarrow\mathbb{Z}$ where $f\left(n\right)=n^2+1$.
    \item $f:\mathbb{Z}\longrightarrow\mathbb{Z}$ where $f\left(x\right)=\begin{cases}\frac{n}{2} & \text{if $n$ is even} \\ 0 & \text{if $n$ is odd}\end{cases}$.
    \item $f:\mathbb{R}\longrightarrow\mathbb{R}$ where $f\left(x\right)=\begin{cases}\frac{1}{x} & \text{if }x\neq0 \\ 0 & \text{if }x=0\end{cases}$.
    \item $f:\mathbb{N}\longrightarrow\mathbb{N}$ where $f\left(n\right)=\begin{cases}2^n & \text{if $n$ is even} \\ n & \text{if $n$ is odd}\end{cases}$.
    \item $f:2^{\mathbb{Z}}\longrightarrow2^{\mathbb{Z}}$ where $f\left(A\right)=A\cup\left\{0\right\}$.
\end{enumerate}
\begin{solution}

% Type your solution to Problem 1 here.

\end{solution}

%%%%%%%%%%%%%%%%%%%%%%%%%%%%%%%%%%%%%%%%%%%%%%%%%%%%%%%%%%%%%%%%

\newpage

%%%%%%%%%%%%%%%%%%%%%%%%%%%%%%%%%%%%%%%%%%%%%%%%%%%%%%%%%%%%%%%%
%%%%%                       Problem 2                      %%%%%
%%%%%%%%%%%%%%%%%%%%%%%%%%%%%%%%%%%%%%%%%%%%%%%%%%%%%%%%%%%%%%%%

\section*{Problem 2}
Let $A=\left\{1,2,3,4\right\}$ and $B=\left\{5,6,7\right\}$. Let $f$ be the relation $\left\{\left(1,5\right),\left(2,5\right),\left(3,6\right),\left(x,y\right)\right\}$, where the values of $\left(x,y\right)$ are to be filled by you. Give an example of $\left(x,y\right)\in A\times B$ so that
\begin{enumerate}[a)] % The enumerate environment produces a numbered list of items. The [a)] ensures that the items are labelled with letters instead.
    \item The relation $f$ is not a function.
    \item The relation is a function from $A$ to $B$ but not onto $B$.
    \item The relation is a function from $A$ to $B$ and is onto $B$.
\end{enumerate}
\begin{solution}

% Type your solution to Problem 2 here.

\end{solution}

%%%%%%%%%%%%%%%%%%%%%%%%%%%%%%%%%%%%%%%%%%%%%%%%%%%%%%%%%%%%%%%%

\newpage

%%%%%%%%%%%%%%%%%%%%%%%%%%%%%%%%%%%%%%%%%%%%%%%%%%%%%%%%%%%%%%%%
%%%%%                       Problem 3                      %%%%%
%%%%%%%%%%%%%%%%%%%%%%%%%%%%%%%%%%%%%%%%%%%%%%%%%%%%%%%%%%%%%%%%

\section*{Problem 3}
Let $A$ be an $n$-element set and let $i,j,k\in\mathbb{N}$ with $i+j+k=n$. How many functions $f:A\longrightarrow\left\{0,1,2\right\}$ are there for which all three of the below are satisfied:
\begin{itemize}
    \item $\left|\left\{a\in A~\big\vert~f\left(a\right)=0\right\}\right|=i$,
    \item $\left|\left\{a\in A~\big\vert~f\left(a\right)=1\right\}\right|=j$,
    \item $\left|\left\{a\in A~\big\vert~f\left(a\right)=2\right\}\right|=k$.
\end{itemize}
Your answer should be in terms of $n$, $i$, $j$ and $k$.
\begin{solution}

% Type your solution to Problem 3 here.

\end{solution}

%%%%%%%%%%%%%%%%%%%%%%%%%%%%%%%%%%%%%%%%%%%%%%%%%%%%%%%%%%%%%%%%

\newpage

%%%%%%%%%%%%%%%%%%%%%%%%%%%%%%%%%%%%%%%%%%%%%%%%%%%%%%%%%%%%%%%%
%%%%%                       Problem 4                      %%%%%
%%%%%%%%%%%%%%%%%%%%%%%%%%%%%%%%%%%%%%%%%%%%%%%%%%%%%%%%%%%%%%%%

\section*{Problem 4}
You have $20$ jellybeans and you want to eat all of the jellybeans over the course of $2$ weeks. Suppose that you eat at least one jellybean a day. Prove, using the pigeonhole principle, that there is a set of consecutive days where you ate exactly $7$ jellybeans.

\begin{solution}

% Type your solution to Problem 4 here.

\end{solution}

%%%%%%%%%%%%%%%%%%%%%%%%%%%%%%%%%%%%%%%%%%%%%%%%%%%%%%%%%%%%%%%%

\newpage

%%%%%%%%%%%%%%%%%%%%%%%%%%%%%%%%%%%%%%%%%%%%%%%%%%%%%%%%%%%%%%%%
%%%%%                       Problem 5                      %%%%%
%%%%%%%%%%%%%%%%%%%%%%%%%%%%%%%%%%%%%%%%%%%%%%%%%%%%%%%%%%%%%%%%

\section*{Problem 5}
Let $A$ be a set of $10$ distinct integers between $1$ and $100$, inclusive.
\begin{enumerate}[a)] % The enumerate environment produces a numbered list of items. The [a)] ensures that the items are labelled with letters instead.
    \item Use the pigeonhole principle to prove that there are two different non-empty subsets of $A$ such that the sum of their elements are the same.
    \item Prove that there are two non-empty \emph{disjoint} subsets of $A$ such that the sum of their elements is the same.
\end{enumerate}
\begin{solution}

% Type your solution to Problem 5 here.

\end{solution}

%%%%%%%%%%%%%%%%%%%%%%%%%%%%%%%%%%%%%%%%%%%%%%%%%%%%%%%%%%%%%%%%

\newpage

%%%%%%%%%%%%%%%%%%%%%%%%%%%%%%%%%%%%%%%%%%%%%%%%%%%%%%%%%%%%%%%%
%%%%%                       Problem 6                      %%%%%
%%%%%%%%%%%%%%%%%%%%%%%%%%%%%%%%%%%%%%%%%%%%%%%%%%%%%%%%%%%%%%%%

\section*{Problem 6}
Let $A=\left\{x\in\mathbb{Z}~\big\vert~3\mid x\right\}$. Show that $A$ and $\mathbb{N}$ have the same cardinality. \medskip

\emph{Hint: Define a function $f:A\longrightarrow\mathbb{N}$ and show that it is a bijection.}
\begin{solution}

% Type your solution to Problem 6 here.

\end{solution}

%%%%%%%%%%%%%%%%%%%%%%%%%%%%%%%%%%%%%%%%%%%%%%%%%%%%%%%%%%%%%%%%

\newpage

%%%%%%%%%%%%%%%%%%%%%%%%%%%%%%%%%%%%%%%%%%%%%%%%%%%%%%%%%%%%%%%%
%%%%%                       Problem 7                      %%%%%
%%%%%%%%%%%%%%%%%%%%%%%%%%%%%%%%%%%%%%%%%%%%%%%%%%%%%%%%%%%%%%%%

\section*{Problem 7}
Let $f:A\longrightarrow B$ be a function. For any subset $X$ of $A$ we define $$f\left(X\right)=\left\{f\left(x\right)~\big\vert~x\in X\right\}.$$ Let $X$ and $Y$ be subsets of $A$. Prove that $f\left(X\cup Y\right)=f\left(X\right)\cup f\left(Y\right)$.
\begin{solution}

% Type your solution to Problem 7 here.

\end{solution}

%%%%%%%%%%%%%%%%%%%%%%%%%%%%%%%%%%%%%%%%%%%%%%%%%%%%%%%%%%%%%%%%

\newpage

%%%%%%%%%%%%%%%%%%%%%%%%%%%%%%%%%%%%%%%%%%%%%%%%%%%%%%%%%%%%%%%%
%%%%%                       Problem 8                      %%%%%
%%%%%%%%%%%%%%%%%%%%%%%%%%%%%%%%%%%%%%%%%%%%%%%%%%%%%%%%%%%%%%%%

\section*{Problem 8}
Suppose $A$ and $B$ are non-empty sets and $f:A\longrightarrow B$ and $g:B\longrightarrow A$ are functions that satisfy $g\circ f=\text{id}_A$.
\begin{enumerate}[a)] % The enumerate environment produces a numbered list of items. The [a)] ensures that the items are labelled with letters instead.
    \item Is is necessarily true that $f$ is surjective?
    \item Is it necessarily true that $g$ is injective?
    \item Is it necessarily true that $f\circ g=\text{id}_B$?
\end{enumerate}
\begin{solution}

% Type your solution to Problem 8 here.

\end{solution}

%%%%%%%%%%%%%%%%%%%%%%%%%%%%%%%%%%%%%%%%%%%%%%%%%%%%%%%%%%%%%%%%

\end{document}
%
% Anything typed after \end{document} will not be included in the pdf