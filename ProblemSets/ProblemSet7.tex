%%%%%%%%%%%%%%%%%%%%%%%%%%%%%%%%%%%%%%%%%%%%%%%%%%%%%%%%%%%%%%%%
%                                                              %
% This is a LaTeX file.  It is a text file that is compiled    %
% by a program called LaTeX into a pretty PDF file.            %
% If you're viewing this file on Overleaf,                     %
% you'll see that PDF in the window to the right.              %
%                                                              %
% The LaTeX macro language is complicated, so we have inserted %
% lots of documenting comments into the file.  Comments start  %
% with `%' and continue to the end of the line.                %
%                                                              %
% Comments are provided to let you, the student, understand    %
% what that part of the code is doing or to provide you with   %
% instructions.                                                % 
%                                                              %
% Skip anything else you don't understand, or ask me.          %
%                                                              %
%%%%%%%%%%%%%%%%%%%%%%%%%%%%%%%%%%%%%%%%%%%%%%%%%%%%%%%%%%%%%%%%
%
\documentclass{article}
\usepackage[left=1in,right=1in,top=1in,bottom=1in]{geometry}
% 
%%%%%%%%%%%%%%%%%%%%%%%%%%%%%%%%%%%%%%%%%%%%%%%%%%%%%%%%%%%%%%%%
%                                                              %
% This is the preamble of the document. This is where we       %
% declare the packages we need for the pdf file to compile     %
% correctly. Packages contain the different commands we        %
% need to use that allow us to format the document nicely.     %
%                                                              %
% You shouldn't need to edit any of these packages.            %
%                                                              %
%%%%%%%%%%%%%%%%%%%%%%%%%%%%%%%%%%%%%%%%%%%%%%%%%%%%%%%%%%%%%%%%
% 
\usepackage{amsmath, amsthm, amsfonts} % These packages contain most of the commands needed to format the maths symbols.
\usepackage{enumerate} % Package that contains options for the \begin{enumerate} environment.
\usepackage{hyperref} % Package that allows hyperlinks.
\newtheorem*{theorem}{Theorem} % Defines an environment with the heading "Theorem". The * supresses the numbering.
\newtheorem*{claim}{Claim} % Defines an environment with the heading "Claim". The * supresses the numbering.
\newtheorem*{proposition}{Proposition} % Defines an environment with the heading "Proposition". The * supresses the numbering.
\theoremstyle{definition}
\newtheorem*{definition}{Definition} % Defines an environment with the heading "Definition". The * supresses the numbering.
\newenvironment{solution}{\bigskip\hrule{\hfill}}{\bigskip\hrule{\hfill}} % Defines an environment that draws lines to make clear where your solution starts and ends.
\newcommand{\Var}{\text{Var}}
% 
%%%%%%%%%%%%%%%%%%%%%%%%%%%%%%%%%%%%%%%%%%%%%%%%%%%%%%%%%%%%%%%%
%                                                              %
% In the author command below, type in your name. The article  %
% class will produce a title page using the command \maketitle %
% containing the title of the document, the author name and    %
% the date.                                                    %
%                                                              %
%%%%%%%%%%%%%%%%%%%%%%%%%%%%%%%%%%%%%%%%%%%%%%%%%%%%%%%%%%%%%%%%
%
\title{\textbf{MATH-UA 120 Discrete Mathematics: \\ Problem Set 7}}
\author{%
    Elizabeth Bennet % Change to your name!
}
\date{Due Monday, November 25th, 2024} % The due date of the assignment. All assignemets are dues at 11:59pm on the date listed
%
%%%%%%%%%%%%%%%%%%%%%%%%%%%%%%%%%%%%%%%%%%%%%%%%%%%%%%%%%%%%%%%%
%                                                              %
% The body of the document is typed in between the lines       %
% \begin{document} and \end{document}.                         %
%                                                              %
%%%%%%%%%%%%%%%%%%%%%%%%%%%%%%%%%%%%%%%%%%%%%%%%%%%%%%%%%%%%%%%%
%
\begin{document}
\maketitle % This command generate the title page information that was filled in in the preamble

\vfill

% The following are the asignment instructions. You should leave these alone and, after reading them, proceed to the problems.
\section*{Assignment Instructions}

\begin{itemize}
    \item These are to be written up in \LaTeX{} and turned in on Gradescope.
    \item \href{https://bit.ly/3yXgVOl}{\textbf{Click here to duplicate this \texttt{.tex} file in Overleaf}}.
    \item Write your solutions inside the \texttt{solution} environment.
    \item You are always encouraged to talk problems through with your peers and your instructor, but your write up should be done independently.
    \item Problems are graded on correctness and fluency.
    \item Unless stated otherwise, all calculations require justification.
    \item Some tutorials on how to use \LaTeX{} can be found \href{https://www.overleaf.com/learn/latex/Tutorials}{\underline{here}}. If you have any questions about \LaTeX{} commands you can always ask your instructor for advice.
\end{itemize}

\vfill

\section*{Statement on generative AI}

In this and other mathematics courses, you are expected to construct clear and concise mathematical arguments based on statements proven in our text and class notes. Large language models such as ChatGPT are unable to produce this kind of solution. They also frequently generate circular logic and outright false results.
 
You may use AI to summarise content, generate study plans, create problems, or do other study-related activities. You may not ask a chatbot to solve your quiz or homework problems, or do any assessment-related activities.
 
You may use AI tools to edit your grammar and punctuation, but remember that mathematical English is not the same as academic English in other disciplines. 

\vfill

\newpage

\newpage

%%%%%%%%%%%%%%%%%%%%%%%%%%%%%%%%%%%%%%%%%%%%%%%%%%%%%%%%%%%%%%%%
%%%%%                       Problem 1                      %%%%%
%%%%%%%%%%%%%%%%%%%%%%%%%%%%%%%%%%%%%%%%%%%%%%%%%%%%%%%%%%%%%%%%

\section*{Problem 1}
A fair coin is flipped $10$ times.
\begin{enumerate}[a)] % The enumerate environment produces a numbered list of items. The [a)] ensures that the items are labelled with letters instead.
    \item What is the probability that there are an equal number of heads and tails?
    \item What is the probability that the first three flips are heads?
    \item What is the probability that there are an equal number of heads and tails \emph{and} the first three flips are heads?
    \item What is the probability that there are an equal number of heads and tails \emph{or} the first three flips are heads? 
    \begin{itemize}
        \item[] For clarity, when we say $A$ \emph{or} $B$, this always includes the outcome where both events occur.
    \end{itemize}
    \item What is the probability that the first three flips are heads \emph{given} that an equal number of heads and tails are flipped?
\end{enumerate}
\begin{solution}

% Type your solution to Problem 1 here.

\end{solution}

%%%%%%%%%%%%%%%%%%%%%%%%%%%%%%%%%%%%%%%%%%%%%%%%%%%%%%%%%%%%%%%%

\newpage

%%%%%%%%%%%%%%%%%%%%%%%%%%%%%%%%%%%%%%%%%%%%%%%%%%%%%%%%%%%%%%%%
%%%%%                       Problem 2                      %%%%%
%%%%%%%%%%%%%%%%%%%%%%%%%%%%%%%%%%%%%%%%%%%%%%%%%%%%%%%%%%%%%%%%

\section*{Problem 2}

An unfair coin shows heads with probability $p$ and tails with probability $1-p$. Suppose this coin is flipped $2$ times. Let $A$ be the event that the coin comes up heads first and tails second. Let $B$ be the event that the coin comes up tails first and heads second.
\begin{enumerate}[a)] % The enumerate environment produces a numbered list of items. The [a)] ensures that the items are labelled with letters instead.
    \item Find $P\left(A\right)$.
    \item Find $P\left(B\right)$.
    \item Find $P\left(A\mid A\cup B\right)$.
    \item Find $P\left(B\mid A\cup B\right)$.
\end{enumerate}

\begin{solution}

% Type your solution to Problem 2 here.

\end{solution}

%%%%%%%%%%%%%%%%%%%%%%%%%%%%%%%%%%%%%%%%%%%%%%%%%%%%%%%%%%%%%%%%

\newpage

%%%%%%%%%%%%%%%%%%%%%%%%%%%%%%%%%%%%%%%%%%%%%%%%%%%%%%%%%%%%%%%%
%%%%%                       Problem 3                      %%%%%
%%%%%%%%%%%%%%%%%%%%%%%%%%%%%%%%%%%%%%%%%%%%%%%%%%%%%%%%%%%%%%%%

\section*{Problem 3}

Suppose that $A$ and $B$ are events in a sample space $\left(S,P\right)$. Prove or disprove:
\begin{enumerate}[a)] % The enumerate environment produces a numbered list of items. The [a)] ensures that the items are labelled with letters instead.
    \item If $P\left(A\cap B\right)=0$, then $P\left(A\mid B\right)=P\left(B\mid A\right)$ if and only if $P\left(A\right)=P\left(B\right)$.
    \item If $P\left(A\right)>0$ and $P\left(B\right)>0$ but $P\left(A\cap B\right)=0$, then $P\left(A\mid B\right)=P\left(B\mid A\right)$. If proven, give an example of two such events with $P\left(A\right)\neq P\left(B\right)$.
\end{enumerate}

\begin{solution}

% Type your solution to Problem 3 here.

\end{solution}

%%%%%%%%%%%%%%%%%%%%%%%%%%%%%%%%%%%%%%%%%%%%%%%%%%%%%%%%%%%%%%%%

\newpage

%%%%%%%%%%%%%%%%%%%%%%%%%%%%%%%%%%%%%%%%%%%%%%%%%%%%%%%%%%%%%%%%
%%%%%                       Problem 4                      %%%%%
%%%%%%%%%%%%%%%%%%%%%%%%%%%%%%%%%%%%%%%%%%%%%%%%%%%%%%%%%%%%%%%%

\section*{Problem 4}

A multiple choice exam has four options for each of the problems. Suppose a student has studied enough so that:
\begin{itemize}
    \item There is a $50\%$ chance that they know the answer to a problem.
    \item There is a $25\%$ chance that they can eliminate one of the incorrect answers.
    \item There is a $25\%$ chance that they don't know how to solve the problem.
\end{itemize}
If the student knows the answer to a problem they will answer it correct;y. If they do not know the answer they will choose randomly from the three or four options. \medskip

You want the exam to to accurately measure what the students does and does not know, not how lucky they can be at guessing answers.

\begin{enumerate}[a)] % The enumerate environment produces a numbered list of items. The [a)] ensures that the items are labelled with numbers contained in parenthesis.
    \item Calculate the probability that the student knew to eliminate one option and got the answer correct.
    \item Calculate the probability that the student gets the answer correct.
    \item Calculate the probability that the student knew how to do the problem given that they got the answer correct.
\end{enumerate}
\begin{solution}

% Type your solution to Problem 4 here.

\end{solution}

%%%%%%%%%%%%%%%%%%%%%%%%%%%%%%%%%%%%%%%%%%%%%%%%%%%%%%%%%%%%%%%%

\newpage

%%%%%%%%%%%%%%%%%%%%%%%%%%%%%%%%%%%%%%%%%%%%%%%%%%%%%%%%%%%%%%%%
%%%%%                       Problem 5                      %%%%%
%%%%%%%%%%%%%%%%%%%%%%%%%%%%%%%%%%%%%%%%%%%%%%%%%%%%%%%%%%%%%%%%

\section*{Problem 5}

Provide an alternative proof to Proposition $31.7$ using any of the statements in Proposition $31.8$.

\begin{proposition}[31.7]
    Let $A$ and $B$ be events in a sample space $\left(S,P\right)$. Then $$P\left(A\right)+P\left(B\right)=P\left(A\cup B\right)+P\left(A\cap B\right).$$
\end{proposition}
\begin{proposition}[31.8]
    Let $\left(S,P\right)$ be a sample space and let $A$ and $B$ be events. We have the following:
\begin{enumerate}[(1)] % The enumerate environment produces a numbered list of items. The [(1)] ensures that the items are labelled with numbers contained in parenthesis.
        \item If $A\cap B=\emptyset$, then $P\left(A\cup B\right)=P\left(A\right)+P\left(B\right)$.
        \item $P\left(A\cup B\right)\leq P\left(A\right)+P\left(B\right)$.
        \item $P\left(S\right)=1$.
        \item $P\left(\emptyset\right)=0$.
        \item $P\left(\overline{A}\right)=1-P\left(A\right)$.
    \end{enumerate}
\end{proposition}
\begin{solution}

% Type your solution to Problem 5 here.

\end{solution}

%%%%%%%%%%%%%%%%%%%%%%%%%%%%%%%%%%%%%%%%%%%%%%%%%%%%%%%%%%%%%%%%

\newpage

%%%%%%%%%%%%%%%%%%%%%%%%%%%%%%%%%%%%%%%%%%%%%%%%%%%%%%%%%%%%%%%%
%%%%%                       Problem 6                      %%%%%
%%%%%%%%%%%%%%%%%%%%%%%%%%%%%%%%%%%%%%%%%%%%%%%%%%%%%%%%%%%%%%%%

\section*{Problem 6}

Consider the sample space $S=\left\{a,b,c\right\}$ with equal probability for each outcome. Define the random variables $X$ and $Y$ by the values in the table below. 
\begin{center}
    \begin{tabular}{c|ccc}
            & $ a$ & $b$ & $c$ \\\hline
        $X$ & $-1$ & $0$ & $1$ \\
        $Y$ & $ 0$ & $1$ & $0$ \\
    \end{tabular}
\end{center}
Check that $\Var\left(X+y\right)=\Var\left(X\right)+\Var\left(Y\right)$ but that $X$ and $Y$ are not independent.
\begin{solution}

% Type your solution to Problem 6 here.

\end{solution}

%%%%%%%%%%%%%%%%%%%%%%%%%%%%%%%%%%%%%%%%%%%%%%%%%%%%%%%%%%%%%%%%

\newpage

%%%%%%%%%%%%%%%%%%%%%%%%%%%%%%%%%%%%%%%%%%%%%%%%%%%%%%%%%%%%%%%%
%%%%%                       Problem 7                      %%%%%
%%%%%%%%%%%%%%%%%%%%%%%%%%%%%%%%%%%%%%%%%%%%%%%%%%%%%%%%%%%%%%%%

\section*{Problem 7}

Suppose we have $n$ identical tiles numbered $1,2,\dots n$ in a bag and we draw all of the tiles out, one at a time, without replacement. Let $X$ be a random variable that gives the number of tiles for which the number on the tile is larger than the number on the tiles drawn before it. \medskip

For example, if $n=4$ and we drew the tiles in the order $4,2,1,3$, then $X=1$ because only the first tile's number is larger than all of tile numbers before it. If we drew the tiles in the order $3,1,2,4$, then $X=2$ because $3$ and $4$ have numbers larger than all of the ones that came before them. \medskip

Prove that $E\left(X\right)=1+\frac{1}{2}+\frac{1}{3}+\cdots+\frac{1}{n}$. \medskip

\emph{Hint: Write $X$ as the sum of simpler random variables.}
\begin{solution}

% Type your solution to Problem 7 here.

\end{solution}

%%%%%%%%%%%%%%%%%%%%%%%%%%%%%%%%%%%%%%%%%%%%%%%%%%%%%%%%%%%%%%%%

\newpage

%%%%%%%%%%%%%%%%%%%%%%%%%%%%%%%%%%%%%%%%%%%%%%%%%%%%%%%%%%%%%%%%
%%%%%                       Problem 8                      %%%%%
%%%%%%%%%%%%%%%%%%%%%%%%%%%%%%%%%%%%%%%%%%%%%%%%%%%%%%%%%%%%%%%%

\section*{Problem 8}

\begin{enumerate}[a)] % The enumerate environment produces a numbered list of items. The [a)] ensures that the items are labelled with letters instead.
    \item Let $\left(S,P\right)$ be a sample space and let $A\subseteq S$ be an event. Define a random vairable $I_A$ whose value at $s\in S$ is $$I_A\left(s\right)=\begin{cases}1 & \text{if }s\in A,\text{ and} \\ 0 & \text{otherwise.}\end{cases}$$ The random variable $I_A$ is called an \emph{indicator} random variable because its value indicates whether or not an event occurred. 

    Prove that $E\left(I_A\right)=P\left(A\right)$.

    \item Let $\left(S,P\right)$ be a sample space and let $X:S\longrightarrow\mathbf{N}$ be a non-negative-integer-valued random variable. Let $a$ be a positive integer. Prove that $$P\left(X\geq a\right)\leq\frac{E\left(X\right)}{a}.$$
\end{enumerate}
\begin{solution}

% Type your solution to Problem 8 here.

\end{solution}

%%%%%%%%%%%%%%%%%%%%%%%%%%%%%%%%%%%%%%%%%%%%%%%%%%%%%%%%%%%%%%%%

\end{document}
%
% Anything typed after \end{document} will not be included in the pdf